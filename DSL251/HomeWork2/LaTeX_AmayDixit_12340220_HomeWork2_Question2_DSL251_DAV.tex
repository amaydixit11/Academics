\documentclass[12pt]{article}
\usepackage{amsmath, amssymb, geometry}
\usepackage{graphicx}
\usepackage{listings}
\usepackage{amsmath}
\usepackage{listings}
\usepackage{xcolor} 
\usepackage{fontenc}
\usepackage{float}
\usepackage{hyperref}
\geometry{a4paper, margin=1in}

% Document metadata
\title{DSL251 \\ Data Analytics and Visualization \\ Homework 2}
\author{Amay Dixit \\ 12340220}
\date{}

\begin{document}

\maketitle

\begin{center}
\href{https://colab.research.google.com/drive/1SiAznpjyErzbzoyA8r2L4zOqpUfBzYaz?usp=sharing}{\texttt{Google Colab Notebook Link}}
https://colab.research.google.com/drive/1SiAznpjyErzbzoyA8r2L4zOqpUfBzYaz?usp=sharing
\end{center}

\section*{Solution to Question 5.5}

We are given the following functions:

\[
f_1(x) = \sin(x_1) \cos(x_2), \quad x \in \mathbb{R}^2
\]
\[
f_2(x, y) = x^\top y, \quad x, y \in \mathbb{R}^n
\]
\[
f_3(x) = xx^\top, \quad x \in \mathbb{R}^n
\]

\subsection*{(a) Dimensions of $\frac{\partial f_i}{\partial x}$:}

\textbf{For \( f_1(x) \):}

The function \( f_1(x) = \sin(x_1) \cos(x_2) \) is a scalar function, where \( x_1, x_2 \in \mathbb{R} \). We compute the partial derivatives:

\[
\frac{\partial f_1}{\partial x_1} = \cos(x_1) \cos(x_2), \quad \frac{\partial f_1}{\partial x_2} = -\sin(x_1) \sin(x_2)
\]
Thus, the Jacobian \( J_1 \) is:
\[
J_1 = \begin{bmatrix} \frac{\partial f_1}{\partial x_1} & \frac{\partial f_1}{\partial x_2} \end{bmatrix} = \begin{bmatrix} \cos(x_1) \cos(x_2) & -\sin(x_1) \sin(x_2) \end{bmatrix} \in \mathbb{R}^{1 \times 2}.
\]

\textbf{For \( f_2(x, y) \):}

The function \( f_2(x, y) = x^\top y \) is a scalar function, where \( x, y \in \mathbb{R}^n \). We compute the partial derivatives:

\[
\frac{\partial f_2}{\partial x} = y^\top \quad \text{and} \quad \frac{\partial f_2}{\partial y} = x^\top
\]
Thus, the Jacobian \( J_2 \) is:
\[
J_2 = \begin{bmatrix} \frac{\partial f_2}{\partial x} & \frac{\partial f_2}{\partial y} \end{bmatrix} = \begin{bmatrix} y^\top & x^\top \end{bmatrix} \in \mathbb{R}^{1 \times 2n}.
\]

\textbf{For \( f_3(x) \):}

The function \( f_3(x) = xx^\top \) is a matrix, where \( x \in \mathbb{R}^n \). The dimensions of the Jacobian depend on the components of \( f_3(x) \). We compute the partial derivatives with respect to each \( x_i \):

\[
\frac{\partial f_3}{\partial x_1} = \begin{bmatrix} x_1 \\ x_2 \\ \vdots \\ x_n \end{bmatrix} \begin{bmatrix} x_1 & 0 & \cdots & 0 \end{bmatrix}, \quad \frac{\partial f_3}{\partial x_2} = \begin{bmatrix} 0 \\ x_1 \\ \vdots \\ x_n \end{bmatrix} \begin{bmatrix} 0 & x_2 & \cdots & 0 \end{bmatrix}, \quad \text{and so on.}
\]
Thus, the Jacobian matrix \( J_3 \) is:
\[
J_3 = \begin{bmatrix} \frac{\partial f_3}{\partial x_1} & \frac{\partial f_3}{\partial x_2} & \cdots & \frac{\partial f_3}{\partial x_n} \end{bmatrix} \in \mathbb{R}^{n^2 \times n}.
\]

\subsection*{(b) Compute the Jacobians:}

\textbf{For \( f_1(x) \):}

We already computed the Jacobian matrix \( J_1 \) in part (a) as:
\[
J_1 = \begin{bmatrix} \cos(x_1) \cos(x_2) & -\sin(x_1) \sin(x_2) \end{bmatrix} \in \mathbb{R}^{1 \times 2}.
\]

\textbf{For \( f_2(x, y) \):}

We already computed the Jacobian \( J_2 \) in part (a) as:
\[
J_2 = \begin{bmatrix} y^\top & x^\top \end{bmatrix} \in \mathbb{R}^{1 \times 2n}.
\]

\textbf{For \( f_3(x) \):}

The Jacobian matrix \( J_3 \) is the matrix of partial derivatives of each component \( f_3(x) \) with respect to each element of \( x \). Each component \( f_3(x) \) is the outer product \( xx^\top \), and the derivative with respect to each \( x_i \) is a matrix with all zeros except for a column where \( x_i \) is repeated. Therefore, the Jacobian is:
\[
J_3 = \begin{bmatrix} \frac{\partial f_3}{\partial x_1} & \frac{\partial f_3}{\partial x_2} & \cdots & \frac{\partial f_3}{\partial x_n} \end{bmatrix} \in \mathbb{R}^{n^2 \times n}.
\]


\section*{Solution to Question 5.8}

\textbf{Part (a):}

We are given the function:
\[
f(z) = \exp\left(-\frac{1}{2} z\right), \quad z = g(y) = y^\top S^{-1} y, \quad y = h(x) = x - \mu
\]
where \(x, \mu \in \mathbb{R}^D\), \(S \in \mathbb{R}^{D \times D}\).

Using the chain rule, the derivative of \(f\) with respect to \(x\) is:
\[
\frac{df}{dx} = \frac{df}{dz} \cdot \frac{dz}{dy} \cdot \frac{dy}{dx}
\]
First, we compute each partial derivative:

1. \(\frac{df}{dz} = \frac{d}{dz} \exp\left(-\frac{1}{2} z\right) = -\frac{1}{2} \exp\left(-\frac{1}{2} z\right)\), with dimensions \(1 \times 1\).

2. \(z = g(y) = y^\top S^{-1} y\), so:
\[
\frac{dz}{dy} = \frac{d}{dy} \left( y^\top S^{-1} y \right) = 2 S^{-1} y, \quad \text{with dimensions} \quad D \times D.
\]

3. \(y = h(x) = x - \mu\), so:
\[
\frac{dy}{dx} = \frac{d}{dx} (x - \mu) = I_D, \quad \text{with dimensions} \quad D \times D.
\]

Now, combining these terms:
\[
\frac{df}{dx} = -\frac{1}{2} \exp\left(-\frac{1}{2} z\right) \cdot 2 S^{-1} y \cdot I_D = - \exp\left(-\frac{1}{2} z\right) S^{-1} y
\]
Thus, the final derivative is:
\[
\frac{df}{dx} = - \exp\left(-\frac{1}{2} z\right) S^{-1} y, \quad \text{with dimensions} \quad D \times 1.
\]

\textbf{Part (b):}

We are given the function:
\[
f(x) = \text{tr}(xx^\top + \sigma^2 I), \quad x \in \mathbb{R}^D.
\]
Here, \(\text{tr}(A)\) represents the trace of matrix \(A\), which is the sum of the diagonal elements \(A_{ii}\).

We first compute the derivative of the trace term:
\[
\frac{d}{dx} \text{tr}(xx^\top) = \frac{d}{dx} \left( \sum_{i=1}^D \sum_{j=1}^D x_i x_j \right) = 2x
\]
Thus:
\[
\frac{df}{dx} = 2x.
\]

\textbf{Part (c):}

We are given the function:
\[
f = \tanh(z) \in \mathbb{R}^M, \quad z = Ax + b, \quad x \in \mathbb{R}^N, \quad A \in \mathbb{R}^{M \times N}, \quad b \in \mathbb{R}^M.
\]
The derivative of \(f\) with respect to \(x\) is:
\[
\frac{df}{dx} = \frac{df}{dz} \cdot \frac{dz}{dx}
\]
First, we compute each partial derivative:

1. \(\frac{df}{dz} = \frac{d}{dz} \tanh(z) = 1 - \tanh^2(z)\), with dimensions \(M \times M\).

2. \(\frac{dz}{dx} = \frac{d}{dx}(Ax + b) = A\), with dimensions \(M \times N\).

Now, combining these terms:
\[
\frac{df}{dx} = (1 - \tanh^2(z)) \cdot A
\]
Thus, the final derivative is:
\[
\frac{df}{dx} = (1 - \tanh^2(z)) \cdot A, \quad \text{with dimensions} \quad M \times N.
\]

\end{document}